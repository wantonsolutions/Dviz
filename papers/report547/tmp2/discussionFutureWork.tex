%%%%%%%%%%%%%%%%%%%%%%%%%%%%%%%%%%%%%%%%%%%%%
\section{Discussion and Future Work}
\label{sec:dafw}
%%%%%%%%%%%%%%%%%%%%%%%%%%%%%%%%%%%%%%%%%%%%%

\noindent\textbf{limitations}
            Our approach to visualizing distributed state is limited
            by computational power. Execution times of t-SNE grow
            polynomial with the length of a trace, therefore traces
            with lengths of tens of thousands would execute slowly
            regardless of parallelism. One heavy handed approach to
            gain scalability would be to spread out t-SNE computation
            on a cluster. However, the size of the trace would have to
            be in the tens of thousands to overcome the cost of
            synchronizing state.

            Logged state has a limited view of a programs behaviour.
            All state collected with our tracing technique must be at
            the application layer and not hidden within binaries. The
            more state which is logged, the better our clustering
            technique responds. Therefore, we are limited to
            applications where the majority of functionality, and
            interesting behaviour is resident in users application
            code. One potential solution to this problem would be to
            analyze a programs stack and heap at the OS layer. While
            this would provide a more holistic view of a programs
            state during execution it sacrifices important information
            such as variable names, log lines, not to mention a state
            space explosion.

            Allowing users to weight variables manually introduces the
            possibility of bias and error into our state model. Users
            with little experience of the programs they trace may be
            biased towards variables which are inconsequential, and
            could lead to their misunderstanding of a programs
            execution. This restricts our users to those which have a
            comprehensive knowledge of their software, and are in
            search of anomalies and bugs.

\noindent\textbf{future work}
            Clusters generated by t-SNE form a de facto state machine
            when connected temporally. Future work could extend our
            visualization to abstract clusters away, and present a
            state machine, where transitions between clusters were
            labeled with variables which caused them. The state of
            individual clusters could be simply encoded with their
            unique invariants.
        %
            Our current visualization is limited to a single
            execution. This approach may be useful for traces known
            to contain bugs, and abnormalities, but is arguably poor
            for checking subtle differences between multiple
            executions. Future work could cluster 2 or more executions
            together and compare their clustering transitions for
            similarities and differences. 
