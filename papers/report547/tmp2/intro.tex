%%%%%%%%%%%%%%%%%%%%%%%%%%%%%%%%%%%%%%%%%%%%%
\section{Introduction and Motivation}
\label{sec:intro}
%%%%%%%%%%%%%%%%%%%%%%%%%%%%%%%%%%%%%%%%%%%%%


The complexities of distributed systems have long plagued
their developers. Writing code which executes on various machines is
intrinsically more complicated due to networking eccentricities, such as
failures, partitions, and message delays. Debugging and checking the
correctness of such systems is laborious and technical, as developers
must inspect large logs for small discrepancies in expected values, and
timestamps. At scale auxiliary tools are necessary for interpreting logs
and making them accessible. Typically tracing tools are used to
reconstruct the communication of nodes throughout a system, order their
events, and present developers with a comprehensive view of an
execution.
%%
    Tracing tools alone still produce large amounts of data, albeit
their structure is more understandable than raw logs. Typically tracing
tools are equipped with a visual front end, allowing users to quickly
observe the behaviour of executions, and under scrutiny identify
irregular or bugging behaviour. No one tracing technique is sufficient
for debugging distributed systems. While all tracing tools are
concerned with debugging they typically fall into 3 sub categories of
performance tuning, distributed control flow, and model checking. Each
of these tracing objectives have different flavors of visualization
which pair with them.
%%

Performance tracing tools such
as~\cite{tensorflow2015-whitepaper,36356,202574,Nagel96vampir:visualization,Zaki:1999:TSP:1080598.1080606},
aim to decrease latency of distributed operations by maximizing
concurrency, and minimizing time spent blocking. Performance
visualizations often aligned parallel processes approximately, and color
encode large latency's to help developers identify sub optimal
segments of their executions.

Control flow tracing tools such
as~\cite{DBLP:conf:icse:2016c,AbrahamsonBBE2014,5071901} help
developers debug fine grain messaging protocols. Visualizations of
such traces must often encode all messages, as developers must reason
about all potential message orderings. A common encoding for
distributed control flow is a message passing graph, in which time is
encoded as a vertical line for each traced host, and messages are
directed arrows connecting host lines.

Model checking trace tools like
~\cite{Beschastnikh:2014:IMC:2568225:2568246,Barham03magpie:online,Walker98visualizingdynamic}
help developers verify the correctness of their system by generating an
abstract model of an execution. Such systems typically capture control
flow, and state transitions of a trace, and compose a finite state
machine (FSM) of a systems execution. Visualizations of such traces
often encode the trace as a control flow points (such as the entrance
of a function) connected with directed edges to other control flow
points. These visuals can grow to an incomprehensible scale as the
size of many models is exponential.

We propose a tracing tool which captures state similar to a model
checking trace tools, with the exception that rather than logging
control flow, our tracing tool only logs a distributed programs state.
Such tracing is unconventional and does not fit nicely into the
aforementioned tracing categories. As such our unique requirements
demand innovative visualization solutions. Entirely state based program
analysis has ties in the world of trajectory
programming~\cite{Waterland:2014:AAS:2654822.2541985,181250,Waterland:2013:CC:2485732.2485749}.
In trajectory analysis simple ML techniques such as weather man, and
mean prediction, as well as linear, and logistic regression are used
to automatically predict and parrallelize computation based solely on
state analysis.

Our proposed visualization for traces of program state applies a similar ML
technique, t-SNE clustering wich is a dimensionality reduction
algorithm\todo{cite tsne}. we leverage t-SNE to clusters points in a programs
execution based on the similarities of a programs state.  T-SNE requires a
distancing function to cluster state, our distance function is as follows. The
distance between two trace points $p$ and $p'$, whose state is composed of an
identical set of variables with potentially different values $V \in \{v_1, v_2,
\dots v_m\}$ calculate the XOR of each pair of variables. Each one bit in the
resultant XOR is a difference of 1bit between the variables. We calculate the
difference between two variables as the number of one bits in the XOR. The
distance between trace points $p$ and $p'$ is the euclidean norm of all
variable distances.

We found the results of our initial technique promising, and the high level
behaviour of distributed programs we traced. However, our visualization suffers
due to a high level of computational complexity in running t-SNE. A single step
of the iterative t-SNE algorithm is $O(n)$, and the algorithm is typically
executed for greater that 20 iterations before a reasonable clustering is
obtained. Typical runtimes for traces consisting of 80 - 100 trace points,
containing 80 - 100 variables each resulted in runtimes in the tens of minutes.
This significant barrier to interactivity lead us to alter the architecture of
our tool, and implement parallel t-SNE to achieve interactive sub 10s
computation times. \todo{talk about variable reweighting as an interactive
technique} \todo{talk about why determining the cause of two distant points is dificult} \todo{talk about slow javascript}

An additional feature of our tracing tool is the analysis of
distributed data invariants. We infer invariants using Dinv, a
distributed front end for Daikon~\cite{Ernst99dynamicallydiscovering}.
These invariants are inferred over entire traces of an execution, and
the number of invariants can be large and incomprehensible (~300-400).
In addition to our improvements of t-SNE we refined our invariant
analysis by first, logically detecting t-SNE clusters using k-Means
clustering, deriving per cluster invariants, and refining those
further to unique invariants for each cluster. This processing step
greatly reduces spurious, and uninteresting invariants, and profiles
cluster behaviour more precisely.

The rest of the paper is layed out as follows.
Section~\ref{sec:related} covers related work. Section~\ref{sec:scope}
overviews the scope of our contributions, and Section~\ref{sec:imp}
covers our implemntation. Section~\ref{sec:res} details our
preformance results and data refinement accuracy. Finally
Section~\ref{sec:dafw} and Section~\ref{sec:conclusion} denote
potential future work, and the conclustion of the paper.
