%%%%%%%%%%%%%%%%%%%%%%%%%%%%%%%%%%%%%%%%%%%%%
\section{Introduction and Motivation}
\label{sec:intro}
%%%%%%%%%%%%%%%%%%%%%%%%%%%%%%%%%%%%%%%%%%%%%

Developing distributed systems is a difficult task. Inherent
concurrency, and non-determinism complicate understanding how a system
behaves. Developers lack tools for providing insight about the state
of a system during it's execution.  The lack of insight makes triaging
bugs an arduous task involving the manual inspection of multiple logs.
Visualization is useful for quickly articulating information.  \dviz
is a visualization tool for distributed systems. \dviz uses logs of
state and time to generate an approximate FSM mined from a systems
execution. States of the FSM are generated by differentiating the
state and plotting states on a plane which clusters similar states.
State of the FSM are linked via a time curve, which connects all
states linearly. Users learn about states by examining concrete
variable values at individual points along the curve. This
implementation only approximates an FSM and does not provide a query
interface for the rich data, such as data invariants which are readily
available. We propose that \dviz's utility can be greatly improved by
adding a flexible query language and by partitioning the time curve
into a labeled FSM.
