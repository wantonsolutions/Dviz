\section{Background}
\label{sec:background}


\noindent{\textbf{Distributed Snapshot}} is an algorithm which
proposes that consistent distributed state can be captured without
interfering with the execution of a system itself
~\cite{dist_snapshots_Chandy1985}.  Distributed snapshots can be
computed online or mined from a log containing vector clocks which
provide a partial ordering of events in a
system~\cite{mattern_vector_clocks_1989}. We consider distributed
snapshots to be a fundamental granularity for examining consistent
state. Our state analysis technique is therefore applied at the level
of a distributed snapshot.

\noindent{\textbf{\dinv}} is a tool which detects likely data
invariants in distributed systems~\cite{dinv}. \dinv operates by
instrumenting distributed systems to log system state and vector
clocks.  Execution logs from the nodes of the system are merged
together, and the state of the system is reconstructed and output as a
distributed system trace. \dinv leverages Daikon to automatically
infer data invariants on the trace.  We plan to use \dinv as a tool
for capturing distributed state.

\noindent{\textbf{\dviz}} is a visualization tool which plots
distributed snapshots mined by \dinv onto a 2D plane. The position of
each snapshot is determined by a XOR distance function. The distance
between all states are computed and each state is plotted so that a
the triangular inequality holds for all states. t-SNE clustering is
used to compute snaphot position~\cite{maaten2008visualizing}. Each
point is linked by a time curve~\cite{Bach2015timecurves}.Each point
is also color coded based on their temporal ordering. The first
snapshot is colored bright red, and the final snapshot is dark brown.
All intermediate points are colored with linear interpolated
luminosity. Figure~\ref{fig:put-get-curve} is a \dviz plot generated
from the execution of the etcd key value store~\cite{etcdraft}. The
plot was generated from a test execution of 50 put requests followed
by 50 get requests. Put requests compose the bright red cluster, while
the dark brown corresponds to Get requests. Initial and final states
are encoded by encircling them in blue. In its current implementation
\dviz has no facilities for automatically labling clusters, making
their significance a mystery to users. Further salient information
about the similarites and differnce of culster is only atainably
through the manual inspection of individual states. Labeling each
cluster with cluster specifc information, and allowing users to query
features of the graph would help to contextualize the visualization.

\begin{figure}[h]
    \includegraphics[width=\linewidth]{fig/put-get-curve}%
    \caption{Time curve, with 2 clusters, generated from an etcd raft cluster processing 50 put requests, followed by 50 get requests.\label{fig:put-get-curve}}%
\end{figure}
