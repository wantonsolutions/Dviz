\section{Proposed Approach}
\label{sec:proposed-approach}

To visualize the state of distributed systems we propose an $x$ step solution.
Firstly a system must be instrumented to produce a log of its state. The log of
the system must then me processed to extract consistent snapshots of the
execution. The differential of consecutive snapshots are calculated. The
differential of each node is visualized. Each of these steps is detailed below.
To instrument distributed systems we will make use of \dinv~\cite{dinv} and
GoVector~\cite{govector}. \dinv will be used to log state, and Govector will
log vector clocks. \dinv will also be used to calculate consistent cuts from a
systems logs. To perform differential analysis of each system we will construct
a difference function $diff$. $diff$ will compute the differential between the
variables of consecutive distributed states. The state differential will be a
vector of $m$ variables corresponding to the velocity of distributed state
transition. We perform differential analysis on the state velocity function and
visualize it.
