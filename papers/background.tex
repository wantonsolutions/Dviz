\section{Background}
\label{sec:background}

\textbf{Distributed State}

\begin{itemize}
    \item \textbf{distributed snapshot}

The distribute snapshot algorithm proposes that consistent distributed state
can be captured without interfering with the execution of a system itself
~\cite{dist_snapshots_Chandy1985}. Distributed snapshots can be computed online
or mined from a log containing vector clocks which provide a partial ordering of events in a system~\cite{mattern_vector_clocks_1989}.

    \item \textbf{\dinv}

\dinv is a tool which detects likely data invariants in distributed
systems~\cite{dinv}. \dinv operates by instrumenting distributed systems to log
state and vector clocks. Execution logs from the nodes of the system are merged
together, and the state of the system is reconstructed and output as a
distributed system trace. \dinv leverages Daikon to automatically infer data
invariants on the trace.

    \item \textbf{state transition}

\textit{I have no idea what to write here, I could not find anything}

\end{itemize}


\textbf{Distributed Visualization}

\begin{itemize}

\item modular visualization of distributed systems

\item pip \textit{http://issg.cs.duke.edu/pip/}

\item Overview: A Framework for Generic Online Visualization of Distributed
Systems

\item Jumpshot \textit{http://www.mcs.anl.gov/research/projects/perfvis/software/viewers/}

\item vampire: http://citeseer.ist.psu.edu/viewdoc/summary?doi=10.1.1.38.1615

\end{itemize}
